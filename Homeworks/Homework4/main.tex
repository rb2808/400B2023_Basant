\documentclass{article}
\usepackage[utf8]{inputenc}
\usepackage[a4paper, total={6in, 8in}]{geometry}

\title{Homework - 4 Astronomy 400 B}
\author{Ritvik Basant}
\date{09 February 2023}

\begin{document}

\maketitle

%\section{Galaxy Component Masses - Table}

%\begin{table}[ht]
%    \centering
%    \begin{tabular}{cccccc}
%    \hline
%        Galaxy Name & Halo Mass & Disk Mass & Bulge Mass & Total Mass & $f_{baryonic}$\\
%        & ($10^{12} M_{\odot}$) & ($10^{12} M_{\odot}$) & ($10^{12} M_{\odot}$) & ($10^{12} M_{\odot}$) & \\
%        \hline
%        Milky Way & 1.975 & 0.075 & 0.010 & 2.060 & 0.041 \\
%       M31 & 1.921 & 0.120 & 0.019 & 2.060 & 0.067 \\
%        M33 & 0.187 & 0.009 & 0.000 & 0.196 & 0.046 \\
%        \hline
%        Local Group & 4.083 & 0.204 & 0.029 & 4.316 & 0.054\\
%        \hline
%    \end{tabular}
%    \caption{This table summarizes the calculated quantities for all three galaxies and the local group.}
%    \label{tab:datap}
%\end{table}

\section{Questions}
\textbf{1.What is the COM position (in kpc) and velocity (in km/s) vector for the MW, M31
and M33 at Snapshot 0 using Disk Particles only (use 0.1 kpc as the tolerance so we
can have the same answers to compare)? In practice, disk particles work the best for
COM determination. Recall that the MW COM should be close to the origin of
the coordinate system (0,0,0).}	\\

\noindent For Milky Way, (i) the distance vector is $(-2.07, 2.95, -1.45)$ kpc and the magnitude is $3.885$ kpc, and (ii) the velocity vector is $(0.94, 6.32, -1.35)$ km/s and the magnitude is $6.531$ km/s. For M31, (i) the distance vector is $(-377.66, 611.43, -284.64)$ kpc and the magnitude is $772.977$ kpc, and (ii) the velocity vector is $(72.85, -72.14, 49.0)$ km/s and the magnitude is $113.632$ km/s. Lastly, for M33, (i) the distance vector is $(-476.22,  491.44, -412.4)$ kpc and the magnitude is $798.982$ kpc, and (ii) the velocity vector is $(44.42, 101.78, 142.23)$ km/s and the magnitude is $180.449$ km/s.\\


\noindent \textbf{2. What is the magnitude of the current separation (in kpc) and velocity (in km/s) between the MW and M31? Round your answers to three decimal places. From class,
you already know what the relative separation and velocity should roughly be (Lecture2
Handouts; Jan 16).} \\

\noindent The magnitude of the current separation between Milky Way and M31 galaxy is $769.098$ kpc. The magnitude of the relative velocity between these two galaxies is $117.738$ km/s. Both values agree with the ones from the lecture.\\

\noindent \textbf{3. What is the magnitude of the current separation (in kpc) and velocity (in km/s) between M33 and M31? Round your answers to three decimal places.}	 \\

\noindent The magnitude of the current separation between M33 and M31 galaxy is $201.083$ kpc. The magnitude of the relative velocity between these two galaxies is $199.370$ km/s.\\

\noindent \textbf{4. Given that M31 and the MW are about to merge, why is the iterative process to determine the COM is important?}	 

\noindent The iterative process to determine the COM for both galaxies is important to analyze the merger as by iteratively calculating the center of mass positions, we will be able to ignore the tidally ripped stars - those stars which are far far away from the concentrated part of the galaxy and thus not play an important role in the analysis of the merger. 


\end{document}
