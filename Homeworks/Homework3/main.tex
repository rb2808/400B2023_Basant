\documentclass{article}
\usepackage[utf8]{inputenc}
\usepackage[a4paper, total={6in, 8in}]{geometry}

\title{Homework - 3 Astronomy 400 B}
\author{Ritvik Basant}
\date{February 2023}

\begin{document}

\maketitle

\section{Galaxy Component Masses - Table}

\begin{table}[ht]
    \centering
    \begin{tabular}{cccccc}
    \hline
        Galaxy Name & Halo Mass & Disk Mass & Bulge Mass & Total Mass & $f_{baryonic}$\\
         & ($10^{12} M_{\odot}$) & ($10^{12} M_{\odot}$) & ($10^{12} M_{\odot}$) & ($10^{12} M_{\odot}$) & \\
        \hline
        Milky Way & 1.975 & 0.075 & 0.010 & 2.060 & 0.041 \\
        M31 & 1.921 & 0.120 & 0.019 & 2.060 & 0.067 \\
        M33 & 0.187 & 0.009 & 0.000 & 0.196 & 0.046 \\
        \hline
        Local Group & 4.083 & 0.204 & 0.029 & 4.316 & 0.054\\
        \hline
    \end{tabular}
    \caption{This table summarizes the calculated quantities for all three galaxies and the local group.}
    \label{tab:datap}
\end{table}

\section{Questions}
\textbf{1.}	The total mass for both Milky Way and M31 galaxies are equal. The dark matter component of both galaxies dominates their total mass. \\

\noindent \textbf{2.}	The total stellar mass for MW is 0.085×1012 solMass while for M31 is 0.139×1012 solMass. So, the ratio of stellar mass for MW/M31 is (0.085 / 0.139) = ~ 0.612. As the hotter blue stars dominate the luminosity from a galaxy, and these blue stars are found in the disk, we should expect M31 to be more luminous as the disk stellar mass for M31 is greater than that of MW. \\

\noindent \textbf{3.}	The dark matter mass for MW is 1.975×1012 solMass while that for M31 is 1.921×1012 solMass. The ratio for dark matter mass for MW/M31 is 1.028, i.e., roughly the same. This is somewhat surprising as both the galaxies have same total mass and roughly the same dark matter mass so they should have the same stellar mass, which they don’t. \\

\noindent \textbf{4.}	 The calculated baryonic fraction for the three galaxies is: Milky Way – 0.041, M31 – 0.067 and M33 – 0.046. Lastly, the baryonic fraction for the Local Group is 0.054. Now, the given baryonic fraction for the universe is 0.16, which is way greater than that calculated for the three galaxies and the local group. I believe that majority of the mass is inhabited between galaxies in the form of WHIM (warm-hot intergalactic medium). I think that this takes the majority mass and if is included in the calculations then the baryonic mass fraction increase. 


\end{document}
